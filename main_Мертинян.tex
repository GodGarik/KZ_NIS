\documentclass[a4paper,14pt]{article}
\usepackage{times}
\usepackage[utf8]{inputenc}
\usepackage[russian]{babel}

% \usepackage{fontspec}
% \setmainfont{Times New Roman}

\setlength{\parindent}{0pt} % Убираем красную строку

\usepackage{geometry} % Настроить поля

\geometry{
    a4paper, % Размер бумаги (A4)
    top=3cm, % Верхний отступ
    bottom=3cm, % Нижний отступ
    left=3cm, % Левый отступ
    right=3cm, % Правый отступ
}



\usepackage[english,russian]{babel}
\title{\begin{center} Обзор литературы \end{center} \newline "Кредитный скоринг и его разработка" }

\author{Мертинян Игорь}
\date{БЭАД 223}
\newpage
\begin{document}

\maketitle

\section*{Введение}
Кредитный скоринг – это ключевой инструмент в финансовой сфере, определяющий вероятность платежеспособности заемщика. Его разработка представляет собой сложный процесс, основанный на анализе множества данных и использовании современных методов аналитики и машинного обучения. В современном мире, где финансовые решения становятся все более автоматизированными, разработка эффективных моделей кредитного скоринга играет важнейшую роль в обеспечении финансовой устойчивости и безопасности.

\newpage
\section*{Статья №1}
\section*{"Построение скоринговых карт с использованием модели логистической регрессии"}
\subsubsection*{Автор: Сорокин Александр Сергеевич}
\subsubsection*{Год: 2014}
\vspace{14pt}
\subsubsection*{\underline{Оценка содержания}:}
Статья помогает разобраться в том, что такое кредитный скоринг, логистическая регрессия и как два этих понятия связаны. Автор показывает, как пользоваться скоринговыми картами, исследует характеристики заёмщиков и приводит примеры построения скоринговых моделей \newline
\textbf{Положительные аспекты:} \newline
- Хорошо структурированная работа \newline
- Показаны все формулы и расчёты \newline
- Использование различных наборов данных для тестирования \newline
\subsubsection*{\underline{Оценка результатов}:}
\textbf{Положительные аспекты:} \newline
- Автор приводит примеры данных и, используя логистическую регрессию и формулы, приведённые в этой статье, показывает, как можно построить скоринговую карту \newline
\textbf{Отрицательные аспекты:} \newline
- Автор ограничен в методах, поэтому примеры рассматриваются одинаково, нет резонансных случаев \newline
\subsubsection*{\underline{Сравнение результатов}:}
Сорокин Александр Сергеевич рассказывает, что такое скоринговая модель и как можно построить скоринговую карту, используя логистическую регрессию. Он показывает методические подходы к формированию и исследованию характеристик заёмщика и последующему преобразованию их, для успешного построения модели
\subsubsection*{\underline{Ссылка на статью} :}
https://cyberleninka.ru/article/n/postroenie-skoringovyh-kart-s-ispolzovaniem-modeli-logisticheskoy-regressii/viewer


\newpage

\section*{Статья №2}
\section*{"Перспективы внедрения современных технологий искусственного интеллекта в скоринговые системы"}
\subsubsection*{Автор:Абдуллаев Некруз Абдуллаевич}
\subsubsection*{Год: 2023}
\vspace{14pt}
\subsubsection*{\underline{Оценка содержания}:}
- Автор ставит чёткие задачи и приводит методы их выполнения \newline
- Присутствует широкий обзор литературы на темы, посвящённые к теме исследования\newline
- Автор показал, почему использование искусственного интеллекта будет целесообразно в управлении кредитными рисками и банковском менджменте в целом. \newline
- Пристуствуют конкретные примеры, как можно внидрить искусственный интеллект в кредитный скоринг\newline
\subsubsection*{\underline{Оценка результатов}:}
\textbf{Положительные аспекты:} \newline
- Присуствуют как преимущества исскусвенного ителлекта в банковском регулировании, так и недостатки \newline
- Автор аргументирует, почему в процессе кредитования искусственный  интеллект и дальнейшее его обучение, чтобы убрать недостатки, ускорит и сделает анализ и оценку моделей более достоверными. \newline 
\textbf{Отрицательные аспекты:} \newline
- Автор просто приводит свои предположения, то есть "на словах" аргументирует, почему искусственный интеллект поможет в анализе и построении модели. Однако ни одна модель не была приведена. Чёткого примера, где исследование с использованием искусственного интелекта оказалось более быстрым и достоверным, нет \newline
- У статьи нет актуальности, так как банковская система, например, России уже имеет широкое применение автоматизированной скоринговой системой, поэтму данная статья имеет актуальность только в определённых странах с устаревшей банковской системой.  \newline

\subsubsection*{\underline{Сравнение результатов}:}
Автор рассматривает современную литературу и аргументирует, почему внедрение искусственного интеллекта в кредитный скоринг поможет быстрее и более качественно оценивать заёмщика.

\subsubsection*{\underline{Ссылка на статью} :}
https://cyberleninka.ru/article/n/postroenie-skoringovyh-kart-s-ispolzovaniem-modeli-logisticheskoy-regressii/viewer


\newpage

\section*{Статья №3}
\section*{"Теоритические аспекты совершенствования моделирования кредитного скоринга в деятельности коммерческих банковских учреждений"}
\subsubsection*{Авторы: Новосёлова Н.Н., Романов В.А., Хубулова В.В.}
\subsubsection*{Год: 2022}
\subsubsection*{\underline{Оценка содержания}:}
Статья проанализировала деятельность ООО "Хоум кредит энд Финанс Банк" и выявила факторы, которые препятствуют эффективному кредитованию заёмщиков. Были проанализированы характеристики заёмщиков и предложены методы их решения. 
\subsubsection*{\underline{Оценка результатов}:}
\textbf{Положительные аспекты:} \newline
- Представлены различные факторы, которые препятствуют эффективному кредитованию выбранным банком \newline
-Представлены различные решения данных факторов\newline

\textbf{Отрицательные аспекты:} \newline
- Аргументы, приведённые в статье, рассматриваются ан определённой базе людей, соответственно данные сильно расходятся с реальностью \newline
- Авторы не показали, как предложенные ими решения помогут выбранному банку более эффективно давайть кредиты заёмщикам \newline
- Авторы дают рекомендации, не подкрепляя их ни примерами из жизни, ни литературой, посвящённой такой же проблеме, ни расчётами, то есть это просто советы
\subsubsection*{\underline{Сравнение результатов}:}
В данной статье приводятся причины, из-за которых ООО "Хоум кредит энд Финанс Банк" не может эффективно выдавать кредиты заёмщикам, и решения этих проблем. Однако, эти решения ничем не подкреплены, и авторы не доказывают, что предложенные ими решения реально помогут

\subsubsection*{\underline{Ссылка на статью} :}
https://cyberleninka.ru/article/n/teoreticheskie-aspekty-sovershenstvovaniya-modelirovaniya-kreditnogo-skoringa-v-deyatelnosti-kommercheskih-bankovskih-uchrezhdeniy


\newpage

\section*{Статья №4}
\section*{"Сравнение моделей кредитного скоринга на базе методов решающих деревьев"}
\subsubsection*{Автор: Стадников А.О.}
\subsubsection*{Год: 2022}
\subsubsection*{\underline{Замечание к автору и журналу}:}
В названии данной статьи опечатка
\subsubsection*{\underline{Оценка содержания}:}
В данной статье автор приводит сравнение таких методов, как: SVM, Logistic Reg, CART, Random Forest, AdaBoost, CatBoost, XGBoost, LightGBM - и показывает их преимущества и недостатки на разных базах данных.
\subsubsection*{\underline{Оценка результатов}:}
\textbf{Положительные аспекты:} \newline
- Приведены таблицы с рачётами для сравнения различных методов построения скоринговой карты \newline
- Предоставлены базы данных \newline
- Все действия, выполеннные автором, последовательно и чётко описываются в статье \newline
\textbf{Отрицательные аспекты:} \newline
- Расммотрены две базы данных, соответственно показаны преимущества не всех методов, описанных в статье \newline
- Отсутствует описание самого построения скоринговой модели \newline
\subsubsection*{\underline{Сравнение результатов}:}
Автор проанализировал различные методы построения скоринговой модели и сделал вывод, что самым лучшим из представленных им методов, оказался LightGBM.

\subsubsection*{\underline{Ссылка на статью} :}
https://cyberleninka.ru/article/n/sravnenie-modeley-kreditnogo-skoringa-na-baze-metodov-reshayuschih-derevev





\newpage
\section*{Статья №5}
\section*{"Разработка скоринговой модели оценки кредитного риска"}
\subsubsection*{Автор: Нозимов Элдор Анварович}
\subsubsection*{Год: 2022}
\subsubsection*{\underline{Оценка содержания}:}
В статье исследовалась кредитоспособность заёмщиков на основе клиентской базы банка и была построена скоринговая модель, основываясь на полученных данных
\subsubsection*{\underline{Оценка результатов}:}
\textbf{Положительные аспекты:} \newline
- Чётко описанный алгоритм действий для построения скоринговой карты \newline
- Приведены таблицы с результатами \newline
\textbf{Отрицательные аспекты:} \newline
- Отсутствие расчётов и формул/методов, по которым строилась скоринговая карта \newline
- Отсутствие теоретического подтверждения, что все расчёты верны \newline
- Отсутствие практической значимости статьи \newline

\subsubsection*{\underline{Ссылка на статью} :}
https://cyberleninka.ru/article/n/razrabotka-skoringovoy-modeli-otsenki-kreditnogo-riska-1




\newpage

\section*{Статья №6}
\section*{"Методика оценки срока просрочки для решения задачи кластеризации заёмщиков микрофинансовых организаций"}
\subsubsection*{Авторы: Досмухамедов Булат Рамильевич, Кузнецова Валентина Юрьевна}
\subsubsection*{Год: 2021}
\subsubsection*{\underline{Оценка содержания}:}
В статье авторы предлагют оценивать заёмщиков у микрофинансивых организаций не по их платёжеспособности, а по оценке срока просрочки потенциальных заёмщиков.
\subsubsection*{\underline{Оценка результатов}:}
\textbf{Положительные аспекты:} \newline
- Показаны все расчёты \newline
- Приведены графики для визуализации результатов\newline
- Сделан обзор литературы по похожим темам\newline
- Показана экспертная оценка, заявленная в начале статьи\newline
- Данный способ оценки работает без классификации заёмщиков на "хороших" и "плохих"
\textbf{Отрицательные аспекты:} \newline
- Нет информации про экспертов, следовательно экспертное мнение не валидно \newline
- Нет сравнения данного способа оценивания с методом, использующимся сегодня \newline
\subsubsection*{\underline{Сравнение результатов}:}
Авторы предлагают другой метод оценки заёмщиков, который упрощает работу банку, так как данный метод работает для любого количества и "качества" заёмщика, однако для этого нужно экспертное мнение.

\subsubsection*{\underline{Ссылка на статью} :}
https://cyberleninka.ru/article/n/metodika-otsenki-sroka-prosrochki-dlya-resheniya-zadachi-klasterizatsii-zaemschikov-mikrofinansovyh-organizatsiy/viewer



\newpage


\section*{Статья №7}
\section*{"Алгоритм бинарной классификации на основе графов принятия решений в задчах кредитного скоринга"}
\subsubsection*{Автор: Кисляков А.Н.}
\subsubsection*{Год: 2021}
\subsubsection*{\underline{Оценка содержания}:}
В статье авторы рассматривают проблему построения графов принятия решений и исследуют качетво классификационных моделей на их основе
\subsubsection*{\underline{Оценка результатов}:}
\textbf{Положительные аспекты:} \newline
- Показаны все расчёты \newline
- Приведены графики для визуализации результатов\newline
- Проведён анализ результатов моделирования, а также оценка точности \newline
- Сделаны чёткие выводы из проделанного исследования
\textbf{Отрицательные аспекты:} \newline
- Усечённый набор данных \newline
\subsubsection*{\underline{Сравнение результатов}:}
Авторы провели исследование и оценили работу алгоритмов бинарной классификации, сделав выводы: увидели преимущества и недостатки своего алгоритма и предложили решение
\subsubsection*{\underline{Ссылка на статью} :}
https://cyberleninka.ru/article/n/algoritm-binarnoy-klassifikatsii-na-osnove-grafov-prinyatiya-resheniy-v-zadachah-kreditnogo-skoringa


\newpage


\section*{Статья №8}
\section*{"Применение скоринговых моделей при оценке кредитоспособности потенциального заёмщтка"}
\subsubsection*{Авторы: Тимошенко Н.В., Хаджиев М.Р., Еремина Н.В.}
\subsubsection*{Год: 2021}
\subsubsection*{\underline{Оценка содержания}:}
Авторы провели исследование и сравнили разные модели: выявили их недостатки и преимущества, а также проанализировали финансовые и нефинансовые показатели скоринговой модели, разбили заёмщиков на классы по кредитоспосбности и также провели анализ важных для скоринговой карты коэффициентов\newline
\subsubsection*{\underline{Оценка результатов}:}
\textbf{Положительные аспекты:} \newline
- Сформулированы все значимые для понимания термины \newline
- Последовательное и чёткое объяснение каждого шага исследования \newline
- Построены все таблицы для визуализации результата и проведён их аналази: сделаны выводы \newline
- Пристутсвуют все формулы и расчёты, по которым вёлся анализ \newline
\textbf{Отрицательные аспекты:} \newline
- Заёмщиков разбили на классы по одному параметру, что ухудшает выборку
\subsubsection*{\underline{Сравнение результатов}:}
При анализе разных моделей предъявлены все вычисления и сделаны выводы, однако не сказано какие случаи будут резонансными и какие надо сделать действия, чтобы их устранить
\subsubsection*{\underline{Ссылка на статью} :}
https://cyberleninka.ru/article/n/primenenie-skoringovyh-modeley-pri-otsenke-kreditosposobnosti-potentsialnogo-zaemschika



\newpage


\section*{Статья №9}
\section*{"Скоринговая оценка кредитоспособности физических лиц"}
\subsubsection*{Авторы: Симонов П.М., Збоев Е.В.}
\subsubsection*{Год: 2020}
\subsubsection*{\underline{Оценка содержания}:}
Автор рассказывает, как строятся скоринговые модели в общем виде, приводит все термины, формулы и вычисления, а также показывает степень влияния факторов на целевую переменную
\subsubsection*{\underline{Оценка результатов}:}
\textbf{Положительные аспекты:} \newline
- В статье объяснены все термины, поэтому любой человек сможет понять, о чём пишет автор \newline
- Показаны формулы и вычисления важных параметров \newline
- Построены таблицы и графики для визуализации результатов \newline
\textbf{Отрицательные аспекты:} \newline
- Нет показателей, по которым сравниваются алгоритмы для построения скоринговой модели \newline
- При изменении данных может получиться, что выводы автора становятся неверными
\subsubsection*{\underline{Сравнение результатов}:}
Автор сравнивает различные алгоритмы и приходит к явным выводам, однако при изменении данных, эти выводы могут стать неврными
\subsubsection*{\underline{Ссылка на статью} :}
https://cyberleninka.ru/article/n/skoringovaya-otsenka-kreditosposobnosti-fizicheskih-lits

\newpage



\section*{Статья №10}
\section*{"An ensemble credit scoring model based on logistic regression with heterogeneous balancing and weighting effects"}
\subsubsection*{Авторы: Zhang Runchi, Xue Ligio, Wang Qin}
\subsubsection*{Год: 2023}
\subsubsection*{\underline{Оценка содержания}:}
Авторы предлагают новую модель Logistic-BWE для построения скоринговой карты, известная как логистическая модель с балансировкой весов, и описывают её преимущества
\subsubsection*{\underline{Оценка результатов}:}
\textbf{Положительные аспекты:} \newline
- Чёткое описание модели. \newline
- Разработан более гибкий метод с динамическим эффектом взвешивания \newline
- Достигнута взаимодоплняемость между подмоделями для построения скоринговой карты \newline
- Рассматриваются способы преодоления проблем анализа реальных статистических данных

\subsubsection*{\underline{Сравнение результатов}:}
- Авторы не только предлагают новую модель Logistic-BWE, но и аргументируют своё мнение, показывая её преимущества. Также они разработали методы новые методы, которые дополняют уже существующие модели
\subsubsection*{\underline{Ссылка на статью} :}
https://www.sciencedirect.com/science/article/abs/pii/S0957417422017511


\newpage



\section*{Статья №11}
\section*{"Сравнение методов калибровки скоринговой модели при прогнозировании логистической регрессией"}
\subsubsection*{Авторы: Широбокова М.А., Лётчиков А.В.}
\subsubsection*{Год: 2017}
\subsubsection*{\underline{Оценка содержания}:}
Авторы выделяют несколько способов калибровки:\newline
-Линейная калибровка от значений вероятностей\newline
-Линейная калибровка от значений шансов\newline
-Логарифмическая калибровка от значений шансов\newline
Затем они посчитали определённые коэффициенты, и сделали выводы, что с помощью данных калибровок, можно прийти к данным, более точно соответвующим реальности
\subsubsection*{\underline{Оценка результатов}:}
\textbf{Положительные аспекты:} \newline
- Чёткие и последовательные действия \newline
- Все данные проиллюстрированны в таблице для визуализации \newline
- Приведены все формулы, используемые в расчётах \newline
- Присутствуют графики выводов исследования \newline
\textbf{Отрицательные аспекты:} \newline
- Отсутствует проверка на переобучаемость модели \newline
\subsubsection*{\underline{Сравнение результатов}:}
Авторы делают вывод, что калибровка изначальных данных может помочь в построении более точной скоринговой карты, однако они не делают проверку, что после использования калибровки, модель не переобучится
\subsubsection*{\underline{Ссылка на статью} :}
https://cyberleninka.ru/article/n/sravnenie-metodov-kalibrovki-skoringovoy-modeli-pri-prognozirovanii-logisticheskoy-regressiey


\newpage



\section*{Статья №12}
\section*{"Мониторинг скоринговой модели в розничном кредитовании"}
\subsubsection*{Автор: Тканко Е.М.}
\subsubsection*{Год: 2019}
\subsubsection*{\underline{Оценка содержания}:}
В представленной статье описан мониторинг работы скоринговой модели \newline
\textbf{Положительные аспекты:} \newline
- Последовательное объяснение, как банки принимают решение о кредитовании \newline
- Чётко выделены критерии, по которым идёт мониторинг \newline
- На графиках показана разница распределений по опредеённым критериям, что упрощает понимание статьи \newline
- Описаны чёткие выводы \newline
- Присутствие конкретных примеров\newline
\textbf{Отрицательные аспекты:} \newline
- Не изучены резонансные случаи \newline
\subsubsection*{\underline{Сравнение результатов}:}
В данной статье авторы чётко показывают, почему мониторинг скоринговой модели важен, для анализа сдвига в распределении входящего потока по скоринговому баллу.

\subsubsection*{\underline{Ссылка на статью} :}
https://cyberleninka.ru/article/n/monitoring-skoringovoy-modeli-v-roznichnom-kreditovanii/viewer

\end{document}
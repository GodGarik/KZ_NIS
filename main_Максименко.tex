\documentclass[a4paper,14pt]{article}
\usepackage[utf8]{inputenc}
\usepackage[english,russian]{babel}

\usepackage{times} % для шрифта times new roman

\setlength{\parindent}{0pt} % Убираем красную строку

\usepackage{geometry} % Настроить поля

\geometry{
    a4paper, % Размер бумаги (A4)
    top=3cm, % Верхний отступ
    bottom=3cm, % Нижний отступ
    left=3cm, % Левый отступ
    right=3cm, % Правый отступ
}




\title{\begin{center} Обзор литературы по теме \end{center} \newline "Построение скоринговой модели" }

\author{Максименко Елизавета}
\date{БЭАД 223}
\newpage
\begin{document}

\maketitle

\section*{Введение}
В современном мире одной из важнейших задач финансовых организаций является определение кредитоспособности клиента. Банкам очень важно минимизировать риски от выдачи кредитов, в связи с чем возникает необходимость выявления вероятности дефолта каждого конкретного субъекта. Для этого финансовые организации разрабатывают скоринговые модели, позволяющие делить заёмщиков на "хороших" и "плохих". \newline
Моей задачей стала разработка такой модели, в связи с чем появилась необходимость изучения доступной литературы из различных источников. Данная работа представляет собой обзор статей, значимых для более полного понимания проблематики и отдельных этапов создания и проверки скоринговых карт. Моей целью на данный момент является выявление недостатков существующих скоринговых моделей и сравнение полученных результатов других исследователей. Также для меня важно выделить положительные аспекты рассматриваемых статей, чтобы в будущем учесть это в своей работе. \newline
Непосредственно выбор темы обусловлен тем, что кредитный скоринг захватывает сразу две сферы: экономическую и машинного обучения. Для меня, как для студента программы "Экономика и анализ данных", объединяющей факультеты экономических и компьютерных наук, это наиболее актуально, поскольку данная работа даёт возможность объединить знания, полученных в ходе обучения. \newline
Теоретическая значимость данного литературного обзора заключается в выявлении недостков существующих моделей и изучении пробелов в исследованиях по этой теме.  \newline
Практическая значимости заключается в дальнейшем применении данного обзора для построения скоринговой модели на примере существующих разработок, но с учётом их недостатков и внедрением новых, ещё не рассмотренных методов.  \newline
Выбирая материал для данного обзора я столкнулась с явным недостатком статей по разработке кредитного скоринга на русском языке за последние годы. Соответственно, новизна моей работы будет заключаться не только в учёте ошибок предыдущих разработчиков, но и в анализе иностранных работ и нацеленности на русскоязычную публику. 

\newpage
\section*{Статья №1}
\section*{"An ensemble credit scoring model based on logistic regression with heterogeneous balancing and weighting effects"}
\subsubsection*{\underline {Авторы}: Zhang Runchi, Xue Liguo, Wang Qin}
\vspace{14pt}
\subsubsection*{\underline{Оценка содержания}:}
Статья представляет собой исследование в области кредитного скоринга. Основная цель статьи - представить новую модель Logistic-BWE, основанную на логистической регрессии с разнообразными эффектами балансировки и взвешивания для улучшения способности модели распознавать дефолтные случаи в условиях дисбаланса выборки. \vspace{1pt} \newline 

\textbf{Положительные аспекты}\textbf{:} \newline
- Чёткое описание модели Logistic-BWE \newline
- Хорошая структурированность \newline
- Использование различных наборов данных для тестирования и сравнения с эталонными моделями. \newline
\subsubsection*{\underline{Оценка результатов}:}
\textbf{Положительные аспекты:} \newline
- Разработка новой модели Logistic-BWE, которая демонстрирует превосходство в распознавании дефолтных случаев и общей точности прогнозирования на различных наборах данных. \vspace{1pt}  \newline
\textbf{Отрицательные аспекты:} \newline
- Ограничения, связанные с применением только одной методологии балансировки \newline
\subsubsection*{\underline{Сравнение результатов}:}
Исследование предоставляет результаты сравнения производительности модели Logistic-BWE с эталонными моделями на различных наборах данных. Оно показывает превосходство Logistic-BWE в распознавании дефолтных случаев и общей точности прогнозирования на большинстве данных. 
\subsubsection*{\underline{Ссылка на статью} :}
https://www.sciencedirect.com/science/article/abs/pii/S0957417422017511


\newpage

\section*{Статья №2}
\section*{"Interpretable machine learning for imbalanced credit scoring datasets"}
\subsubsection*{\underline {Автор}:Yujia Chen}
\vspace{14pt}
\subsubsection*{\underline{Оценка содержания}:}
- Все шаги подкреплены таблицами и схемами для более наглядного понимания \newline
- Присутствует обзор литературы, где представлены работы, использующие различные \newline методы машинного обучения для прогнозирования дефолтов по кредитам \newline
- Важным аспектом было замечание о том, что недостаточно исследований уделяется \newline влиянию дисбаланса классов на методы интерпретации, не привязанные к модели \newline
- Для проверки используются реальные данные из открытых источников, что делает \newline исследование более применимым на практике
\subsubsection*{\underline{Оценка результатов}:}
\textbf{Положительные аспекты:} \newline
- Новизна полученных результатов \newline
- Подтверждено, что дисбаланс класса оказывает негативное влияние на интерпретационную производительность как LIME, так и SHAP \newline
- были введены различные несбалансированные методы обучения для улучшения эффективности классификации \newline 
- Предлагаемая экспериментальная структура может быть использована для исследований в других сферах \vspace{10pt} \newline 
\textbf{Отрицательные аспекты:} \newline
- Когда признаки ранжируются в соответствии с величиной их абсолютного значения \newline коэффициентов, их порядок становится более неопределенным и ближе к случайной \newline перестановке. Таким образом, признаки, занимающие верхние позиции в списках, могут \newline получить более высокое значение согласования ранжирования  \newline
- Методы повторной выборки могут вызвать такие проблемы, как переобучение (методы  \newline избыточной выборки) или потеря информации (методы недостаточной выборки)  \newline
- Методы повторной выборки могут увеличить случайность и добавить шума к исходным входным данным, что нежелательно в финансовых операциях и ограничивает их  \newline распространение в корпоративной среде  \newline

\subsubsection*{\underline{Сравнение результатов}:}
Это первое исследование, которое измеряет стабильность LIME и SHAP с точки зрения классового дисбаланса. Также преимуществом является широкий обзор предшествующих работ и анализ их методологий в контексте кредитного скоринга.

\subsubsection*{\underline{Ссылка на статью} :}
https://www.sciencedirect.com/science/article/pii/S0377221723005088


\newpage

\section*{Статья №3}
\section*{"Feature Selection in a Credit Scoring Model"}
\subsubsection*{\underline {Автор}:Juan Laborda}

\subsubsection*{\underline{Оценка содержания}:}
Статья посвящена методам выбора признаков в моделях кредитного скоринга. Она охватывает различные методы отбора признаков, включая фильтры, обертки и встроенные методы, и применяет их к разным алгоритмам классификации, таким как логистическая регрессия, метод опорных векторов, метод ближайших соседей и случайный лес. 
\subsubsection*{\underline{Оценка результатов}:}
\textbf{Положительные аспекты:} \newline
- Представлены различные методы классификации, что позволяет сравнивать эффективность отбора признаков на различных моделях \newline
- Исследование предлагает методологию, которая может быть использована в практике для улучшения кредитных скоринговых моделей \newline
- Отчетливо демонстрирует важность отбора признаков для улучшения производительности моделей кредитного скоринга. \newline
- Объясняет влияние количества выбранных признаков на производительность моделей \newline
- Методы отбора признаков помогли сделать модели кредитного скоринга более точными \vspace{10pt} \newline
\textbf{Отрицательные аспекты:} \newline
- Необходимо больше исследований на других наборах данных и использование других методов отбора признаков для подтверждения результатов \newline
- Не оценивается влияние времени обучения моделей и других практических аспектов \newline
- Обобщение результатов на другие наборы данных может потребовать дополнительных исследований
\subsubsection*{\underline{Сравнение результатов}:}
В данной статье проводится сравнение различных методов построение модели кредитного скоринга с целью выбора наиболее эффективного. Предоставлена методология выбора ключевых переменных для создания модели оценки кредитоспособности. 
\subsubsection*{\underline{Допущения в исследовании}:}
- В работе использовались классические методы отбора признаков, что может ограничивать обобщение результатов на более сложные датасеты или на применение в других областях, требующих более сложных моделей \newline
- Оценка моделей производилась с использованием MAE, что, хоть и является показательным метрикой, не учитывает все возможные аспекты производительности модели
\subsubsection*{\underline{Ссылка на статью} :}
https://www.mdpi.com/2227-7390/9/7/746


\newpage

\section*{Статья №4}
\section*{"Разработка скоринговой модели оценки кредитного риска"}
\subsubsection*{\underline {Автор}: Нозимов Э.А.}
\subsubsection*{\underline{Оценка содержания}:}
Исследовалась кредитоспосооность заемщиков на основе клиентской базы банка. Было сформировано факторное пространство, позволяющее формализовать и объяснить различия между классами прежних заемшиков. Выявлены факторы, вносяшие наибольший вклад в различение классов. В результате проведения анализа разработана скоринговая молель оценки крелитного риска. позволяющая адекватно и быстро оценить кредитоспособность потенциального клиента банка, что способствует росту конкурентоспособности на рынке кредитования.
\subsubsection*{\underline{Оценка результатов}:}
\textbf{Положительные аспекты:} \newline
- Построенная дискриминантная функция обладает хорошей разделяющей способностью \newline
- Данная функция может быть использована для оценки кредитоспособности клиента \newline
- Предложенная модель может стать эффктивным инструментом снижения рисков \vspace{10pt} \newline
\textbf{Отрицательные аспекты:} \newline
- Недостаточно объяснена суть модели \newline
- Отсутствует описание самого построения скоринговой модели \newline
- Нет проверки модели на конкретных данных \newline
\subsubsection*{\underline{Ссылка на статью} :}
https://cyberleninka.ru/article/n/razrabotka-skoringovoy-modeli-otsenki-kreditnogo-riska-1





\newpage
\section*{Статья №5}
\section*{"Внедрение IT-технологий в процесс формирования кредитного рейтинга заёмщика"}
\subsubsection*{\underline {Авторы}: Морозова Г.В., Жогина К.А.}
\subsubsection*{\underline{Оценка содержания}:}
В статье рассмотрены проблемы внедрения информационных технологий в процесс формирования кредитного ретинга заемщика. исследованы механизмы построения системы кредитного скоринга на основе нейросетевых моделей и их использование в процессе формирования кредитного рейтинга заемщика - физического лица.
\subsubsection*{\underline{Оценка результатов}:}
\textbf{Положительные аспекты:} \newline
- Широкая методологическая база \newline
- Статья объясняет сложности при построении скоринговой модели и работы с ней \newline
- С теоретической точки зрения объяснена необходимость усовершенствования баз данных, предусмотренных для определения кредитного рейтинга заёмщиков \vspace{10pt} \newline
\textbf{Отрицательные аспекты:} \newline
- Отсутствие подтверждения исследованием \newline
- Отсутствие работы с конкретными данными \newline
- Отсутствие практической значимости статьи \newline
- Не предложены способы решения проблем, связанных с работой скоринговых карт \newline

\subsubsection*{\underline{Ссылка на статью} :}
https://cyberleninka.ru/article/n/vnedrenie-it-tehnologiy-v-protsess-formirovaniya-kreditnogo-reytinga-zayomschika



\newpage

\section*{Статья №6}
\section*{"Скоринговые модели как интеллектуальная собственность банка"}
\subsubsection*{\underline {Авторы}: Мешкова Е.И., Данилова Е.Н.}
\subsubsection*{\underline{Оценка содержания}:}
В статье представлена широкая теоретическая база, позволяющая читателю ознакомиться со скоринговыми моделями и особенностями их построения. Объясняется процесс выбора зависимых переменных, формирования выборки, других коэффициентов и показателей. В целом статья может быть полезна для тех, кто хочет разобраться в значении скоринговых моделей для оценки банковских рисков, однако не даёт практических результатов.
\subsubsection*{\underline{Оценка результатов}:}
\textbf{Положительные аспекты:} \newline
- Объяснена выгода банков от скоринговых карт \newline
- Исследованы способы передачи прав на ПО компании, нанявшей разработчика \vspace{10pt} \newline
\textbf{Отрицательные аспекты:} \newline
- Не рассказано о различных методах построения скоринговых карт \newline
- Отсутствует практическая значимость \newline
\subsubsection*{\underline{Сравнение результатов}:}
Скоритговые модели рассмотрены с позиции интеллектуальной собственности. Объяснено, за кем в различных ситуациях закрепляются прва на ПО. 

\subsubsection*{\underline{Ссылка на статью} :}
https://cyberleninka.ru/article/n/skoringovye-modeli-kak-intellektualnaya-sobstvennost-banka



\newpage


\section*{Статья №7}
\section*{"Построение скоринговых карт с использованием модели \newline логистической регрессии"}
\subsubsection*{\underline {Автор}: Сорокин А.С.}

\subsubsection*{\underline{Оценка содержания}:}
Данная статья представляет методику построения скоринговой модели на основе логистической регрессии с использованием различных данных\newline
\textbf{Положительные аспекты:} \newline
- Цели и проблематика чётко изложены. \newline
- Процесс подготовки данных для моделирования подробно описан \newline
- Шаги по построению модели логистической регрессии изложены и подкреплены формулами \newline
- Проведён анализ результатов моделирования, а также оценка точности \vspace{10pt} \newline
\textbf{Отрицательные аспекты:} \newline
- Выбор модели логистической регрессии не обоснован. Указано лишь, что он наиболее часто используется. \newline
- Приведён пример расчета скоринговых баллов лишь для одного атрибута скоринговой карты \newline
- Недостаточная актуальность. Статья была опубликована в 2014 году. \newline
\subsubsection*{\underline{Оценка результатов}:}
Поскольку в статье приведён пример расчета скоринговых баллов лишь для одного атрибута скоринговой карты, невозможно оценить точность результатов, так как как таковых результатов нет. В работе содержится информация о том, как построить скоринговую карту, однако нет приведённого примера скоринговой карты, из-за чего сложно оценить достоверность.
\subsubsection*{\underline{Сравнение результатов}:}
При построении модели логистической регрессии в кредитном скоринге возникает низкая степень точности предсказания отрицательных исходов, т.е. дефолтов.
\subsubsection*{\underline{Ссылка на статью} :}
https://cyberleninka.ru/article/n/postroenie-skoringovyh-kart-s-ispolzovaniem-modeli-logisticheskoy-regressii


\newpage


\section*{Статья №8}
\section*{"Особенности построения скоринговой модели на основе аналитической платформы Deductor"}
\subsubsection*{\underline{Авторы}: Лагерев Д.Г., Бондарева И.В}

\subsubsection*{\underline{Оценка содержания}:}
Данная статья представляет методику построения скоринговой модели на основе аналитической платформы Deductor \newline
\textbf{Положительные аспекты:} \newline
- Сформулированы все значимые для понимания термины \newline
- Широкая теоретическая база \newline
- Перечислены этапы построения модели \newline
- Проведён анализ результатов моделирования, а также оценка точности \vspace{10pt} \newline
\textbf{Отрицательные аспекты:} \newline
- В данной статье построение скоринговой модели на основе аналитической платформы Deductor сравнивается исключительно с построением методом логистической регрессии
\subsubsection*{\underline{Оценка результатов}:}
В ходе исследования было выявлено, что данная методика даёт сравнительно точные результаты. Однако точность зависит от количества данных и точности выбора множества значений.
\subsubsection*{\underline{Сравнение результатов}:}
При построении скоринговой модели на основе аналитической платформы Deductor точность результатов не страдает, однако значительно ускоряется процесс вычислений, делая процесс анализа данных практически автоматическим.
\subsubsection*{\underline{Ссылка на статью} :}
https://cyberleninka.ru/article/n/osobennosti-postroeniya-skoringovoy-modeli-na-osnove-analiticheskoy-platformy-deductor



\newpage


\section*{Статья №9}
\section*{"Разработка скоринговой модели. Методы классификации заемщиков"}
\subsubsection*{\underline{Авторы}: Котляр В.П., Антипова Е.А.}

\subsubsection*{\underline{Оценка содержания}:}
\textbf{Положительные аспекты:} \newline
- В статье рассмотрен процесс построения скоринговой модели «с нуля» \newline
- Описаны примеры подобных методов, с их подробным описанием \newline
- Приведены варианты решения проблемы реализации скоринговой модели на практике, с приведением конкретной компании разработчика в качестве примера \newline
- Приведены все формулы, используемые в расчётах \newline
- Перечислены различные методы построения скоринговой модели с их подробным описанием \vspace{10pt} \newline
\textbf{Отрицательные аспекты:} \newline
- Недостаточная актуальность. Статья была опубликована в 2014 году.
\subsubsection*{\underline{Оценка результатов}:}
В данной статье перечислены и описаны различные методы построения скоринговой модели, однако не присутствует их непосредственное сравнение, что могло бы добавить статье практическую значимость.
\subsubsection*{\underline{Сравнение результатов}:}
Не проведён сравнительный анализ скоринговых моделей
\subsubsection*{\underline{Ссылка на статью} :}
https://cyberleninka.ru/article/n/razrabotka-skoringovoy-modeli-metody-klassifikatsii-zaemschikov


\newpage



\section*{Статья №10}
\section*{"Методологические аспекты построения скоринговых \newline моделей"}
\subsubsection*{\underline {Автор}: Гринь Н.В.}

\subsubsection*{\underline{Оценка содержания}:}
\textbf{Положительные аспекты:} \newline
- Цели и проблематика чётко изложены. \newline
- Сформулированы все значимые для понимания термины \newline
- Широкая теоретическая база \newline
- Рассматриваются способы преодоления проблем анализа реальных статистических данных \vspace{10pt} \newline
\textbf{Отрицательные аспекты:} \newline
- Недостаточная актуальность. Статья была опубликована в 2012 году.
\subsubsection*{\underline{Оценка результатов}:}
Достигнут уровень корректных классификаций 83–88\%

\subsubsection*{\underline{Сравнение результатов}:}
- Для повышения точности классификации на основе скоринговых моделей используется алгоритм корректировки обучающей выборки за счет исключения многомерных аномальных наблюдений и алгоритм формирования области неопределенности в принятии решений на основе композиции алгоритмов статистической классификации, а также способ задания границ области неопределенности на основе анализа апостериорных вероятностей классов. \newline
- Проведён сравнительный анализ скоринговых моделей
\subsubsection*{\underline{Ссылка на статью} :}
https://elib.grsu.by/katalog/176678-405824.pdf


\newpage



\section*{Статья №11}
\section*{"Использование нейросетевых моделей в поведенческом скоринге"}
\subsubsection*{\underline {Авторы}: Сорокин C.В., Сорокин А.С.}
\subsubsection*{\underline{Оценка содержания}:}
\textbf{Положительные аспекты:} \newline
- Цели и проблематика чётко изложены. \newline
- Исходные данные для наглядности приведены в табличном виде \newline
- Приведены все формулы, используемые в расчётах \newline
- Присутствуют графики и зарисовки для объяснения работы модуля нейросетевого анализа данных \newline
- Представлены решения проблем переобучения сети \newline
- Объяснена необходимость разделения данных на выборки \newline
- Приведен пример использования одной из программ — SPSS \vspace{10pt} \newline
\textbf{Отрицательные аспекты:} \newline
- Данный метод влечёт переобучение сети, которое, в свою очередь, несёт значительное увеличение ошибок в дальшнейшей работе \newline
- Недостаточная актуальность. Статья была опубликована в 2015 году.
\subsubsection*{\underline{Оценка результатов}:}
В статье рассмотрена нейросетевая модель является альтернативой практически ставшей стандартом при построении скоринговых систем логистической регрессии. Объяснено, что использование нелинейных функций в узлах нейронной сети позволяет нейронным сетям моделировать сложные нелинейные зависимости. 
\subsubsection*{\underline{Сравнение результатов}:}
Переобучение сети несёт значительное увеличение ошибок, что влечёт дополнительную работу по исправлению этих ошибок
\subsubsection*{\underline{Ссылка на статью} :}
https://cyberleninka.ru/article/n/ispolzovanie-neyrosetevyh-modeley-v-povedencheskom-skoringe


\newpage



\section*{Статья №12}
\section*{"Обзор методов кредитного скоринга"}
\subsubsection*{\underline {Авторы}: Кочеткова В.В., Ефремова К.Д.}
\subsubsection*{\underline{Оценка содержания}:}
В представленной статье рассмотрены основные методы кредитного скоринга  \newline
\textbf{Положительные аспекты:} \newline
- Дано определение кредитного скоринга \newline
- Описана роль кредитных отношений в экономике \newline
- Рассматривается риск невозврата суммы кредита и его связь с оценкой кредитных рисков через кредитный скоринг \newline
- Описаны четыре основных категории оценки кредитных рисков \newline
- Описывается использование математических и статистических методов для разделения кредитных операций на группы риска с целью предсказания возможных невозвратов и нарушений обязательств заемщиков  \newline
- Указывается на разнообразие факторов, влияющих на решение о выдаче кредита \vspace{10pt} \newline
\textbf{Отрицательные аспекты:} \newline
- Отсутствие конкретных примеров  \newline
- Некоторые из описанных методов, такие как нейронные сети или метод опорных векторов, могли бы быть более подробно рассмотрены с упоминанием их особенностей и возможных ограничений

\subsubsection*{\underline{Оценка результатов}:}
Статья предоставляет обзор различных методов кредитного скоринга и упоминает специализированное программное обеспечение, предназначенное для оценки кредитных рисков. Однако, эта статья скорее представляет обзорный материал, чем исследование, поэтому оценка конкретных результатов в контексте новых открытий или экспериментов ограничена.
\subsubsection*{\underline{Сравнение результатов}:}
В данной статье отсутствуют конкретные результаты исследования или анализа эффективности этих методов в реальных условиях
\subsubsection*{\underline{Допущения в статье}:}
- Обобщение описанных методов: В статье представлен обзор различных методов кредитного скоринга, но без конкретных контекстуальных примеров или подробных случаев их применения.  \newline
- Упрощенное представление о программном обеспечении: Предоставление перечня продуктов без детального анализа их функциональности и возможностей.  \newline
- Статья не описывает конкретные сценарии, в которых те или иные методы могут быть более или менее эффективны

\subsubsection*{\underline{Ссылка на статью} :}
https://cyberleninka.ru/article/n/obzor-metodov-kreditnogo-skoringa


\end{document}